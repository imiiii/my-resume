My interest in decision

I am passionate about solving business and marketing-related problems. I enjoy using my algorithm design and programming skills, together with my knowledge in economics and statistics, to translate business problems into mathematical problems that algorithms can solve.
I believe I am a great fit at the Decision Sciences R&D organization at Epsilon for a couple of reasons. First, I genuinely enjoy reading, learning, and research in the domain of human decision-making. I started working in this domain starting from my bachelor studies when I joined a social neuroscience research center and started working on this domain using tools from game theory. 
Second, I believe personalization is going to be a significant problem in machine learning. I spent my Ph.D. working on high-dimension friendly algorithms for customer acquisition and retention in the SaaS business model. I worked on and even designed new algorithms that can personalize marketing interventions to increase customer acquisition and retention.
I believe my skill set and interest in Epsilon's business domain make me a strong candidate for the role. I hope my qualifications are found suitable by the hiring committee as well.


Add to culture

I developed knowledge in various fields throughout my academic and professional life journey, including computer science and engineering, business, statistics, economics, and marketing. I can think like a manager, code like an engineer, design and analyze algorithms like a computer scientist, and think about a research problem like a scientist. I believe my multi-faceted personality could contribute to the culture of Centro in a couple of ways.
First, I can facilitate collaboration between engineers, scientists, and managers. I can help stakeholders translate a business problem into scientific questions that can be solved using data, prototype solutions and collaborate with engineers to develop production-level codes suitable to be added to Centro's framework.
Besides, my diverse background enables me to add to the culture of Centro by fostering a novel interdisciplinary approach toward solving problems. I have practiced this method during my research in my bachelor's and Ph.D. studies. I used game theory and genetic algorithms to understand why human beings are inequity averse in my bachelor thesis. During my Phd, I used the recursive partitioning technique in machine learning literature to develop a new algorithm that solves the curse of dimensionality problem in dynamic discrete choice modeling, a well-known problem in the economics literature.
Finally, my skillsets are driven by my internal curiosity and genuine interest in learning new skills, a habit that still exists in me. This trait would help me to contribute to the culture of Centro by bringing new knowledge, new tools, and new ways of tackling business problems. This quality is crucial for the fast-moving marketing and tech industry environment.